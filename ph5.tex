% Copyright (c) 2015 William Bevington, Callum O'Brien and Alex Pace

% Permission is granted to copy, distribute and/or modify this document
% under the terms of the GNU Free Documentation License, Version 1.3
% or any later version published by the Free Software Foundation;
% with no Invariant Sections, no Front-Cover Texts, and no Back-Cover Texts.

\documentclass{article}

\usepackage{amsmath}

\usepackage{amssymb}

\usepackage{geometry}

\title{PH5}
\author{William Bevington \and Callum O'Brien \and Alex Pace}

\begin{document}

\maketitle
\tableofcontents
\newpage

\section{Electromagnetism}

Electromagnetism is all to do with fields. A field is a load of numbers. Vector fields are the best fields, and all the rest suck, so we only care about vector fields. A uniform field is a field where all the vectors are parallel. In a uniform electric field there is half the potential difference halfway between the poles, that is to say

\[E=V/d\]

\noindent When working with diagrams describing a magnetic field, current traveling into the the page is expressed by a small circle with a cross inside it. Current traveling out of the page can be shown by a small circle witha dot in its centre.

The "right hand rule" (holding one's hand in a "thumbs up" position) can tell the the dirction of a rotating magnetic field (one's fingers) around a wire carrying a cur
rent in a certain direction (your thumb). The "left hand rule" (holding a thumbs up position then pointing the index finger and pointing the middle finger perpendicular to the palm) tells you the direction of the direction of \textbf{m}ovement of a wire (thu\textbf{m}b) with a magnetic \textbf{f}ield (\textbf{f}irst finger) and \textbf{c}urrent (se\textbf{c}ond finger).

The magnitude of an electric field can be defined as the ratio of force to the relevant property. In this case, current and length of conductor. The angle of the wire in the field also affects the force, and hence can be described thus:

\[F=BIL\sin\left(\theta\right)\]

\noindent Unit of B: \(NA^{-1}m^{-1}\) or tesla \(T\)

\subsection{The Magnetic Field of an Infinitely Long Conductor}

Generally,

\[\textrm{d}B = \frac{\mu_0 i \sin\theta \textrm{d}s}{4\pi r^2}\]

\noindent Confining this to a plane perpendicular to the current eliminates $\theta$, giving:

\[\textrm{d}B = \frac{\mu_0 i \textrm{d}s}{4\pi r^2}\]

\noindent Hence,

\[B=\frac{\mu_0}{4\pi}\int_0^\infty\frac{R \textrm{d}s}{\left(s^2 + R^2\right)^{\frac{3}{2}}}=\frac{\mu_0 I}{2\pi R}\left[\frac{S}{\left(s^2 + R^2\right)^{\frac{1}{2}}}\right]_0^\infty=\frac{\mu_0 I}{2\pi R}\]

\noindent Which gives the magnetic flux density of the field created by a current in an infinitely long conductor in a plane perpendicular to the current.

\subsection{Capacitors}

% capacitor circuit diagram symbol

Capacitors are a way to store energy/charge. The energy is stored in the electric field inbetween the two plates.

\[C=\frac{Q}{V}=\frac{\epsilon A}{d}\]
\[W=\frac{1}{2}QV=\frac{1}{2}CV^2\]

\begin{itemize}
	\item $C$ is capacitance
	\item $Q$ is charge
	\item $V$ is potential difference
	\item $\epsilon$ is the permiativity of free space
	\item $A$ is the area of the plates
	\item $d$ is the distance between the plates
	\item $W$ is the energy stored in the field between the plates
\end{itemize}

\noindent For capacitors in parallel;
\[C=\sum_{i=1}^nc_i\]
\noindent For capacitors in series;
\[\frac{1}{C}=\sum_{i=1}^n\frac{1}{c_i}\]

\noindent Regarding discharging capacitors,

\[V=V_0e^{-\frac{t}{RC}}\]
\[Q=Q_0e^{-\frac{t}{RC}}\]
\[I=I_0e^{-\frac{t}{RC}}\]

\subsection{Solenoids}

A solenoid is a coil of wire. Given $n>\frac{1}{\pi}$

\[B=\frac{\mu_0In}{l}\] // this is written in fomula bookelets as just n, not n over l, as that is number of turns per metre

\noindent Where\begin{itemize}

    \item $B$ is field strength / teslas
    \item $I$ is current / amperes
    \item $l$ is the length of the 
    \item $n$ is the number of coils

\end{itemize}

\end{document}
