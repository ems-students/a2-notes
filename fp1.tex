\documentclass{article}

\usepackage{amsmath}
\usepackage{amssymb}

\usepackage{geometry}

\title{Edexcel Advanced Level GCE Mathematics FP1}
\author{William Bevington \and Callum O'Brien \and Alex Pace}
\date{}

\begin{document}

\maketitle
\tableofcontents
\newpage

\section{Matrix Transformations}

\section{Complex Numbers}
                If the complex numbers \(z_1\) and \(z_2\) are equal, then it follows that \(Re\left(z_1\right) = Re\left(z_2\right)\) and \(Im\left(z_1\right) = Im\left(z_2\right)\). as demonstarated bellow: \\\\
                Let \(z_1 = a+bi\) and \(z_2 = c+di\) where \(a,b,c,d, \in \mathbb{R}\)
                \[z_1 = z_2 \therefore a+bi = c+di\]
                \[a-c=\left(d-b\right)i\]
                \[\left(a-c\right)^2=\left(b-d\right)^2i^2 \rightarrow \left(a-c\right)^2=-\left(d-b\right)^2\]
                \[\left(a-c\right)^2\geq 0 \& -\left(d-b\right)^2\leq 0\]
                the only overlap here is 0. \\\\
                therefore, \(Re\left(z_1\right) = Im\left(z_2\right).
                \subsection{Modulus & Argument of Complex Numbers}
                	These values are given when a complex number is represented in the polar form:
                	\[z=r\left(\cos\left(\theta\right) + i\sin\left(\theta\right)\right)\]
                	where \(r\) is the modulus and \(\theta\) is the argument. \\\\
                	From the cartesian form \(z=a+bi\) the modulus and argument of a complex number can be found as follows:
                	\[|z|=\sqrt{a^2 + b^2}\]
                	\[arg\left(z\right) = arctan\left(\frac{b}{a}\right)\]


\end{document}

