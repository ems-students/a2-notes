\documentclass{article}

\usepackage{amsmath}
\usepackage{amssymb}
\usepackage{geometry}

\title{C3}
\author{Callum O'Brien}

\begin{document}

\maketitle
\tableofcontents
\newpage

\section{Partial Fractions}
\subsection{Splitting a Fraction with Two or More Linear Factors in the Denominator}
\[\frac{x+3}{(x+2)(x+1)}=\frac{a}{x+2}+\frac{b}{x+1}=\frac{a(x+1)+b(x+2)}{(x+2)(x+1)}\]
\[\therefore x+3=a(x+1)+b(x+2)\]
Here, two methods can be used; equating coefficients and substitution. Equating coefficients is rather self explanatory, and involves creating simultaneous equations from the fact that coefficients of different powers of $x$ will be equal on both the left- and right-hand-side of the above equation, then solving for A and B;
\[\textit{coefficients of }x\textit{: }1=a+b\]
\[\textit{constants: }3=a+2b\]
\noindent The former rearranges to
\[b=1-a\]
\noindent Substituting this into the latter shows
\[3=a+2(1-a)=2-a\therefore a=-1\Rightarrow b=2\]
\noindent Hence,
\[\frac{x+3}{(x+2)(x+1)}=\frac{2}{x+1}-\frac{1}{x+2}\]
\noindent Substitution involves substituting values for $x$ which neglect one of the unknowns in our equation. In the above example, one would substitute the values $-1$ and $-2$ to neglect the terms containting $a$ and $b$ repectively;
\[x\rightarrow-1\textit{, }b=2\]
\[x\rightarrow-2\textit{, }-a=1\therefore a=-1\]
And once again, we arrive at the same partial fractions,
\[\frac{x+3}{(x+2)(x+1)}=\frac{2}{x+1}-\frac{1}{x+2}\]

\subsection{Splitting a Fraction with a Squared Linear Factor in the Denominator}
\[\frac{}{()^2}=\frac{a}{}+\frac{b}{()^2}=\frac{a()+b}{()^2}\]
\[=a()+b\]

\section{Periodic Functions}

\[a\cos\theta+b\sin\theta=R\cos(\theta+\alpha)=R'\sin(\theta+\alpha')\]
\[a=R\cos\alpha=R'\sin\alpha,b=-R\sin\alpha=R'\cos\alpha\]

\noindent Example:
\[5\cos100\pi x+10\sin100\pi x\]
\[5=R\cos\alpha,10=-R\sin\alpha\]
\[-\frac{10}{5}=\tan\alpha\]
\[\alpha=\arctan-\frac{10}{5}\]

\end{document}
                 
