% Copyright (c) 2015 Callum O'Brien

% Permission is granted to copy, distribute and/or modify this document
% under the terms of the GNU Free Documentation License, Version 1.3
% or any later version published by the Free Software Foundation;
% with no Invariant Sections, no Front-Cover Texts, and no Back-Cover Texts.

\documentclass{article}

\usepackage{amsmath}

\usepackage{amssymb}

\usepackage{amsfonts}

\usepackage{booktabs}

\usepackage{multirow}

\usepackage{geometry}

\usepackage[toc]{glossaries}
\makeglossaries
\newglossaryentry {compkey} {
    name = "composite key",
    description = ""}

\newacronym {DBMS} {DBMS} {Database Management System}

\newglossaryentry {dbms} {
    name = "database management system"
    description = ""}

\newglossaryentry {forekey} {
    name = "foreign key",
    description = ""}

\newglossaryentry {primkey} {
    name = "primary key",
    description = ""}

\newglossaryentry {thrdnorm} {
    name = "third normal form",
    description = ""}



\begin{document}

\title{COMP3}
\author{Callum O'Brien}
\maketitle
\tableofcontents

\newpage

\section{Databases \& DBMS}

A database is collection of non-redundant data that are stored and which the
database management system is designed to manipulate. A database management
system (DBMS) is a software system that enables the definition, creation and
maintenance of a database and provides controlled access to this database.

In a \textit{file-based approach}, these data are stored in a computer file
called a flat file. In such a database, the only way to change the record
structure of these files is to read the file of records using the old file
structure and save the records to a new file using the new record stucture. The
file-based approach has many disadvantages;\begin{enumerate}
    
    \item \textbf{Data inconsistency:} If there are two adresses for one person,
        the error in one is not `spotted.  '

    \item \textbf{Data redundancy:} If someone rents a car more than once their
        details are stored many times

    \item \textbf{Inefficiency:} Typing in the adresses and other details every
        time a car is rented is time-consuming and takes up storage space

    \item \textbf{Difficult to maintain:} As more records are added the database
        may need to query and manipulate large amounts of data. This is more
        difficult using the file-based approach
        
    \item \textbf{Program-data dependence:} Because the file structure has not
        been defined in an accessible way, every computer program has to specify
        exactly what data fields constitute a record in the file being
        processed. If the data structure changes, programs also have to be
        changed to match the new structure.

\end{enumerate} Relational databases provide an advantage over the file-based
approach. Relational databases contain many \textit{tables} that are linked
together. This linking is done by \textit{keys} -- these are common pieces of
information that are shared between tables. In a relational database, the DBMS
creates and maintains the \textit{data dictionary}. \textit{Data definition
language} is used to create tables. It records the attributes, data types, validation used and realtionship between entities. It defines which atttributes
belong to which tables. It also creates users and grants access rights to them.
DDL commands are;\begin{enumerate}

    \item \textbf{CREATE DATABASE}
    
    \item \textbf{CREATE TABLE}
    
    \item \textbf{CREATE USER}
    
    \item \textbf{GRANT}
    
    \item \textbf{DROP}

\end{enumerate}

\textit{Data Manipulation Language} is used to add data to the
database and manipulate those data. DML commands are\begin{enumerate}

    \item \textbf{INSERT}
    
    \item \textbf{UPDATE}

\end{enumerate}

Database schema are different `views' of a database. These can
be \begin{enumerate}

    \item \textbf{External / User Schema:} the way in which each user sees the
        database, there may be several different schema representing each user's
        view.
    
    \item \textbf{Conceptual or Logical Schema:} describes the entities,
        attributes and relationships
    
    \item \textbf{Internal Schema:} describes how the data will be stored and
        how they will be accessed and updated.

\end{enumerate}

\subsection{Entity Relationships}

Entities in a database are tables or users. In an \textit{entity relationship
diagram}, they are represented by rectangles. Relationships between entities are
lines connecting entities. They can be;\begin{enumerate}

    \item One-to-one (e.g. $y=x$)
    
    \item One-to-many (e.g. $y=\pm\sqrt x$)
    
    \item Many-to-one (e.g. $y=\sin x$)
    
    \item Many-to-many (e.g. $y^2=1-x^2$)

\end{enumerate}

\subsection{Database Normalisation}

Normalisation is a technique used to produce a normalised set of entities. This
means that the entities will:\begin{enumerate}

    \item Contain no redundant data

    \item Contain no repeated data

    \item Be able to contain as many or as few tuples as we wish in an entity
        without requiring us to enter data in another entity as well.

\end{enumerate} The database will now be at its most efficient and there will be
no risks of compromising the integrity of the data within it.

Databases that are not normalised to third normal form will have problems with
insertion, deletion and updating anomalies. Database normalisation has been a
well- developed field since the introduction of Codd's work on normal forms.
While it's reasonable easy to identify repeating groups or attributes and remove
these for first normal form, the differences between second and third normal
form are difficult to secure.

There is a further complication in that repeating groups are understood in
different ways by different theorists. As a consequence, there is not universal
agreement as to what constitutes first normal form.

\subsection{Structured Query Language}

SQL is a database language that allows users to create, store, update and
manipulate data in a database. The first version of SQL was developed at IBM by
Chamberlin and Boyce in the early 1970s. It was formally standardised in 1986 by
ANSI.

Almost all modern relational DBMSs use SQL as a standard database language, but
many use dialects of SQL.

SQL has two subsets, DML and DDL. DDL is used to create the
database structure (i.e. to define which attributes belong in which tables). It
also allows you to create users and grant access rights to users.

\begin{table}

    \centering

    \begin{tabular}{ll}

        \toprule

        \multicolumn{2}{c}{DDL} \\

        \midrule

        CREATE DATABASE & Creates a new database \\
        
        ALTER DATABASE  & Modifies a database \\
        
        CREATE TABLE    & Creates a new table \\
        
        ALTER TABLE     & Modifies a table \\

        DROP TABLE      & Deletes a table \\
        
        CREATE INDEX    & Creates an index (search key) \\
        
        DROP INDEX      & Deletes an index \\
        
        GRANT           & Grants permissions on a table \\

        \midrule

        \multicolumn{2}{c}{DML} \\

        \midrule
        
        UPDATE          & Updates data in a database \\
        
        DELETE          & Deletes data from a database \\
        
        INSERT INTO     & Inserts new data into a database \\

        \bottomrule

    \end{tabular}

    \caption{Common DDL and DML commands}

\end{table}

\section{Hashing}

A hash function maps from an input $x$ to an output $y$ in a manner requiring
(relatively) little processing power, but such that $y$ can not be mapped to $x$
without significantly more processing. Hash functions are many-to-one; an ideal
hash function has no inverse. 

A collision occurs when $x_n$ and $x_m$ map to the same $y$. Collisions can be
resolved by mapping $x_m$ to the next available $y$. This approach can be
ineffective, as it may ultimately lead to more collisions.

\end{document}
