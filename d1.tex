\documentclass{article}
\usepackage{amsmath}
\usepackage{amssymb}
\usepackage{geometry}

\title{Graphs and Networks}
\author{Callum O'Brien}
\date{}

\begin{document}

\maketitle
\tableofcontents

\section{Definitions}
A \textbf{graph} consists of points called \textbf{vertices} or \textbf{nodes} which are connected by lines called \textbf{edges} or \textbf{arcs}. If a graph has a number associated with each edge it is a \textbf{network}. Vertices/nodes are referred to by letters, whereas edges are referred to are referred to by two letters.

A subgraph of $G$ is a graph, all of whose vertices belong to $G$. The \textbf{degree} or \textbf{valency} of a vertex is the number of edges incident to it. A \textbf{path} is a finite sequence of edges such that the end node of one edge in the sequence is the start node of the next and in which no node appears more than once. A \textbf{walk} is a path in which some verticies may appears more than once. A \textbf{cycle} or \textbf{circuit} is a closed path, where the finial vertex is the same as the initial one. Two nodes are \textbf{connected} if there exists some path between them. A graph is connected if all its nodes are connected to every other. A \textbf{loop} is an edge which connects a node to itself.

A \textbf{simple} graph has no loops. A \textbf{directed} edge $AB$ connects $A$ to $B$ but not $B$ to $A$. A graph with directed edges is a \textbf{digraph}. A graph which contains no cycles is a \textbf{tree}. If all nodes within a graph share an edge, the graph is \textbf{complete}. In a \textbf{bipartite} graph, there are two groups of nodes with no direct connections between them.

\end{document}
