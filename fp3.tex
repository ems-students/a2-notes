% Copyright (c) 2015 William Bevington, Callum O'Brien and Alex Pace

% Permission is granted to copy, distribute and/or modify this document
% under the terms of the GNU Free Documentation License, Version 1.3
% or any later version published by the Free Software Foundation;
% with no Invariant Sections, no Front-Cover Texts, and no Back-Cover Texts.

\documentclass{article}

\usepackage{amsmath}
\usepackage{amssymb}

\usepackage{array}

\usepackage{geometry}

\usepackage{mathrsfs}

\usepackage{multicol}

\usepackage{tikz}
\usetikzlibrary{arrows}

\newcommand{\de}{\textrm{d}\,}
\newcommand{\st}{\:|\:}
\newcommand{\csch}{\textrm{csch\,}}
\newcommand{\sech}{\textrm{sech\,}}
\newcommand{\arsinh}{\textrm{arsinh\,}}
\newcommand{\arcosh}{\textrm{arcosh\,}}
\newcommand{\artanh}{\textrm{artanh\,}}

\begin{document}

\title{FP3}
\author{William Bevington \and Callum O'Brien \and Alex Pace}
\maketitle
\tableofcontents
\newpage

\section{Hyperbolic Functions}

The hyperbolic functions are analogs of the ordinary trigonometric, or circular
functions. The basic hyperbolic functions are, as one might expect, analygous to
sine and cosine; they are hyperbolic sine and hyperbolic cosine.

\[\sinh : \mathbb{R} \rightarrow \mathbb{R} : x \mapsto \frac{e^x - e^{-x}}{2}\]

\[\cosh : \mathbb{R} \rightarrow \left\{x \st x \in \mathbb{R},\, x \geq
1\right\} : x \mapsto \frac{e^x + e^{-x}}{2}\]

\noindent From these one can derive the hyperbolic tangent, hyperbolic secant,
hyperbolic cosecant and hyperbolic cotangent functions in much the same way as
their circular counterparts.

\[\tanh : \mathbb{R} \rightarrow \left\{x \st x \in \mathbb{R},\, x \in
\left[-1, 1\right]\right\} : x \mapsto \frac{e^x - e^{-x}}{e^x + e^{-x}}\]

\[\sech : \mathbb{R} \rightarrow \left\{y \st y \in \mathbb{R},\, y \in
\left[0,1\right)\right\} : x \mapsto \frac{2}{e^x + e^{-x}}\]

\[\csch : \left\{x \st x \in \mathbb{R},\, x \neq 0 \right\} \rightarrow
\left\{y \st y \in \mathbb{R},\, y \neq 0\right\} : x \mapsto \frac{2}{e^x -
e^{-x}}\]

\[\coth : \left\{x \st x \in \mathbb{R},\, x \neq 0 \right\} \rightarrow
\left\{y \st y \in \mathbb{R},\, y \notin \left[-1, 1\right]\right\} : x \mapsto
\frac{e^x + e^{-x}}{e^x - e^{-x}}\]
\noindent The inverse functions of hyperbolic sine, hyperbolic cosine and hyperbolic tangent are area hyperbolic sine, area hyperbolic cosine and area
hyperbolic tangent;

\[\arsinh : \mathbb{R} \rightarrow \mathbb{R} : x \mapsto \ln\left(x + \sqrt{x^2
+ 1}\right)\]

\[\arcosh : \left\{x \st x \in \mathbb{R},\, x \geq 1 \right\} \rightarrow
\mathbb{R} : x \mapsto \ln\left(x + \sqrt{x^2 - 1}\right)\]

\[\artanh : \left\{x \st x \in \mathbb{R},\, x \in \left[-1, 1\right]\right\}
\rightarrow \mathbb{R} : x \mapsto \frac{1}{2}\ln\left(\frac{x + 1}{1 -
x}\right)\]

\section{Conic Sections}

All conic sections can be described in terms of loci; for any point $P$ on a
conic section,

\[\frac{|PM|}{|PS|} = e\]

\noindent where $e$ is the eccentricity, $S$ is the focus and $M$ is the closest point to
$P$ that lies on the directrix.

\[0 \leq e < 1 \Rightarrow P \textrm{ describes an ellipse}\]
\[e = 1 \Rightarrow P \textrm{ describes a parabola}\]
\[e > 1 \Rightarrow P \textrm{ describes a hyperbola}\]

\subsection{Ellipses}

Ellipses are conic sections with eccentricty $e \in [0,1)$. Considering their
geometry, it is clear that, when expressed as a parametric equation, ellipses
take the form;

\[x = a\cos\theta,\: y = b\sin\theta\]

\noindent Hence;

\[\left(\frac{x}{a}\right)^2 = \cos^2\theta,\: \left(\frac{y}{b}\right)^2 =
\sin^2\theta\]

\noindent Making apparent an ellipse expressed in cartesian form;

\[\left(\frac{x}{a}\right)^2 + \left(\frac{y}{b}\right)^2 = 1\]

\paragraph{Theorem} For $e \in [0,1)$, an ellipse of focus $(ae, 0)$ and
directrix $x = \frac{a}{e}$ is described by the equation;

\[\left(\frac{x}{a}\right)^2 + \left(\frac{y}{b}\right)^2 = 1\]

\paragraph{Proof} 

\[\frac{|PS|^2}{|PM|^2} = e \Rightarrow |PS|^2 = e^2|PM|^2\]

\noindent By pythagoras,

\[|PS|^2 = \left(x - ae\right)^2 + y^2\]

\[|PM|^2 = \left(\frac{a}{e} - x\right)^2 + y^2\]

\noindent Thus

\[\left(x - ae\right)^2 + y^2 = \left(\frac{a}{e} - x\right)^2e^2\]\\

\noindent The derivative of an ellipse can be found by implicitly
differentiating its cartesian form or using the chain rule on its parametric
form. By the former method;

\[\frac{\de}{\de x} \left(\left(\frac{x}{a}\right)^2 +
\left(\frac{y}{b}\right)^2\right) = \frac{\de}{\de x}1\]

\[\frac{2x}{a^2} + \frac{2y}{b^2} \times \frac{\de y}{\de x} = 0\]

\[\frac{\de y}{\de x} = -\frac{b^2x}{a^2y}\]

\noindent And by the latter;

\[\frac{\de x}{\de \theta} = -a\sin\theta\]

\[\frac{\de y}{\de \theta} = b\cos\theta\]

\[\frac{\de y}{\de x} = \frac{\de y}{\de \theta} \times \left(\frac{\de x}{\de
\theta}\right)^{-1} = -\frac{b\cos\theta}{a\sin\theta}\]

\noindent From this we can derive the general equation of a tangent to an
ellipse at a point $\left(i,\, j\right)$:

\[y - j = -\frac{b^2i}{a^2j}\left(x - i\right)\]

\[a^2jy + b^2ix = \left(aj\right)^2 + \left(bi\right)^2\]

\noindent Similarly, we can derive the equation for a normal:

\[y - j = \frac{a^2j}{b^2i}\left(x - i\right)\]

\[b^2iy + a^2ij = a^2jx + b^2ij\]

\end{document}

